\documentclass[9pt]{beamer}
\usetheme{Warsaw}
\usepackage[utf8]{inputenc}
\usepackage[english]{babel}
\usepackage{amsmath}
\usepackage{amsfonts}
\usepackage{amssymb}
\usepackage{graphicx}
\author{leviathanch | chipforge | foshardware \ (Lanceville Technology)}
\title{Breaking the microchip monopoly}
\setbeamercovered{transparent} 
\setbeamertemplate{navigation symbols}{} 
\logo{lsa.png} 
%\institute{} 
%\date{} 
\subject{A free semiconductor manufacturing standard}
\begin{document}

%\begin{frame}
%\tableofcontents
%\end{frame}

\section[Place'n'Route]{}

\begin{frame}{Tools}
	\begin{itemize}
        \setlength\itemsep{1em}
		\item graywolf origins in timberwolf
		\item graywolf simulated annealing
		\item graywolf inline syscalls
		\item qrouter purpose and scope
		\item qrouter sequential routing
		\item qrouter formal correctness, esp libresilicon tech
	\end{itemize}
\end{frame}

\begin{frame}{Tools}
	\begin{itemize}
        \setlength\itemsep{1em}
		\item different tool sets like cadence, alliance, etc
		\item similar capabilities with respect to silicon
		\item open source tools are insufficient, except yosys
	\end{itemize}
\end{frame}

\begin{frame}{State of the Art}
	\begin{itemize}
        \setlength\itemsep{1em}
		\item Place components for a large chip
		\item Route wires roughly along a chessboard for a large chip
		\item Route detailed tracks and vias for a large chip
		\item Formal correctness: Rip-up and Re-route
		\item Formal style: Sequential/Imperative code
	\end{itemize}
\end{frame}

\begin{frame}{Proposed}
	\begin{itemize}
        \setlength\itemsep{1em}
		\item Decomposition for a large chip
		\item Place components and route for small chips in parallel
		\item Place abstract gates and route recursively
		\item Formal correctness: Reduction from SMT
		\item Formal style: Parallel/Functional code
	\end{itemize}
\end{frame}

\section[Decomposition]{}
\begin{frame}{Subcell hierarchies}
	\begin{itemize}
        \setlength\itemsep{1em}
		\item Explicit subcell hierarchies through high modularization
		\item Implicit subcell hierarchies through exlining
		\item Preserve hierarchy in compiler interfaces
	\end{itemize}
\end{frame}

\begin{frame}{High modularization}
	\begin{itemize}
        \setlength\itemsep{1em}
		\item Example of a *very* modular chip
	\end{itemize}
\end{frame}

\begin{frame}{Exlining}
	\begin{itemize}
        \setlength\itemsep{1em}
		\item Proof of concept: picorv
	\end{itemize}
\end{frame}

\section[SMT]{}
\begin{frame}{SMT2}
	\begin{itemize}
        \setlength\itemsep{1em}
		\item Reduction of a *very* common problem and witty problem to SMT
	\end{itemize}
\end{frame}

\begin{frame}{SMT2}
	\begin{itemize}
        \setlength\itemsep{1em}
		\item Show routing related problem in integer programming
	\end{itemize}
\end{frame}

\end{document}
