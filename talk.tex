\documentclass[9pt]{beamer}
\usetheme{Warsaw}
\usepackage[utf8]{inputenc}
\usepackage[english]{babel}
\usepackage{amsmath}
\usepackage{amsfonts}
\usepackage{amssymb}
\usepackage{graphicx}
\usepackage{tikz}
\usetikzlibrary{
	arrows,
	automata,
        shadings,
        shadows,
        shapes,
}

%\usepackage[dvipsnames]{xcolor}
\usepackage{xcolor}

%%  -------------------------------------------------------------------
%%      GDS II layer, regarding MOSIS SCMOS layer map
%%  -------------------------------------------------------------------
% GDS II #41 - P_WELL
\definecolor{pwell}{rgb}{1.0, 0.74, 0.53}   % macaroni and cheese
% GDS II #42 - N_WELL
\definecolor{nwell}{rgb}{0.61, 0.87, 1.0}  % columbia blue
\definecolor{pbase}{rgb}{1.0, 0.51, 0.26}  % mango tango
\definecolor{nbase}{rgb}{0.0, 0.75, 1.0}   % capri 
% GDS II #43 - ACITVE
\definecolor{active}{rgb}{0.9, 0.4, 0.38}   % light carmine pink
% GDS II #45 - N_PLUS_SELECT
\definecolor{nimplant}{rgb}{0.45, 0.76, 0.983}% maya blue
% GDS II #44 - P_PLUS_SELECT
\definecolor{pimplant}{rgb}{1.0, 0.51, 0.26}% mango tango
% GDS II #46 - POLY
\definecolor{poly}{rgb}{0.56, 0.93, 0.56}   % light green
% GDS II #25 - CONTACT
\definecolor{contact}{rgb}{0.83, 0.83, 0.83}% light gray
% GDS II #49 - METAL1
\definecolor{metal1}{rgb}{0.38, 0.31, 0.86} % majorelle blue
% GDS II #50 - VIA1
\definecolor{via1}{rgb}{0.83, 0.83, 0.83}   % light gray
% GDS II #51 - METAL2
\definecolor{metal2}{rgb}{0.04, 0.85, 0.32} % malachite
% GDS II #61 - VIA2
\definecolor{via2}{rgb}{0.83, 0.83, 0.83}   % light gray
% GDS II #63 - METAL3
\definecolor{metal3}{rgb}{0.98, 0.93, 0.37} % maize
% GDS II #30 - VIA3
\definecolor{via3}{rgb}{0.83, 0.83, 0.83}   % light gray
% GDS II #31 - METAL4
\definecolor{metal4}{rgb}{0.75, 0.25, 0.0}  % mahogany
% GDS II #32 - VIA4
\definecolor{via4}{rgb}{0.83, 0.83, 0.83}   % light gray
% GDS II #33 - METAL5
\definecolor{metal5}{rgb}{0.79, 0.08, 0.48} % magenta (dye)
% GDS II #36 - VIA5
\definecolor{via5}{rgb}{0.83, 0.83, 0.83}   % light gray
% GDS II #37 - METAL6
\definecolor{metal6}{rgb}{0.11, 0.35, 0.02} % lincoln green
% GDS II #29 - SILICIDE_BLOCK
\definecolor{silicide-block}{rgb}{0.98, 0.94, 0.9}  % linen
% GDS II #52 - GLASS
\definecolor{glass}{rgb}{1.0, 1.0, 0.88}    % light yellow
% GDS II #26 - PADS
\definecolor{pads}{rgb}{0.75, 1.0, 0.0}     % lime (color wheel)

\definecolor{resist}{rgb}{0.71, 0.4, 0.11}  % light brown

\definecolor{silicide}{rgb}{0.29, 0.33, 0.13}
\definecolor{titanium}{rgb}{0.8, 0.58, 0.46}

\def\OpacityLayout {0.5}

%
% physical
%
\definecolor{substrate}{rgb}{0.96, 0.94, 0.93}  % isabelline
\definecolor{nitride}{rgb}{1.0, 0.03, 0.0}
\definecolor{gateoxide}{rgb}{0.88, 1.0, 1.0}    % light cyan
\definecolor{isolationoxide}{rgb}{0.84, 0.79, 0.87}% languid lavender

\def\CrossSectionOnly{0.3}
\def\CrossAndTopSection{0.2}
\def\CrossAndTopSectionBig{0.3}
\def\VLSILayout{0.4}
\def\UpperContactResist{8.0}
\def\UpperMetalResist{9.0}
\def\UpperMoreMetalResist{16.0}

\def\LowerMetal{4.0}
\def\UpperMetal{4.5}

\def\LowerMoreMetal{5.0}
\def\UpperMoreMetal{5.5}

\def\LowerMoreMetalTwo{6.0}
\def\UpperMoreMetalTwo{6.5}

\def\UpperGlass{7.0}


\author{leviathanch | chipforge | foshardware \ (Lanceville Technology)}
\title{Breaking the microchip monopoly}
\setbeamercovered{transparent} 
\setbeamertemplate{navigation symbols}{} 
\logo{lsa.png} 
%\institute{} 
%\date{} 
\subject{A free semiconductor manufacturing standard}
\begin{document}

%\begin{frame}
%\tableofcontents
%\end{frame}

\section[Place'n'Route]{}

\begin{frame}{Tools}
	\begin{itemize}
        \setlength\itemsep{1em}
		\item graywolf origins in timberwolf
		\item graywolf simulated annealing
		\item graywolf inline syscalls
		\item qrouter purpose and scope
		\item qrouter sequential routing
		\item qrouter formal correctness, esp libresilicon tech
	\end{itemize}
\end{frame}

\begin{frame}{Tools}
	\begin{itemize}
        \setlength\itemsep{1em}
		\item different tool sets like cadence, alliance, etc
		\item similar capabilities with respect to silicon
		\item open source tools are insufficient, except yosys
	\end{itemize}
\end{frame}

\begin{frame}{State of the Art}
	\begin{itemize}
        \setlength\itemsep{1em}
		\item Place components for a large chip
		\item Route wires roughly along a chessboard for a large chip
		\item Route detailed tracks and vias for a large chip
		\item Formal correctness: Rip-up and Re-route
		\item Formal style: Sequential/Imperative code
	\end{itemize}
\end{frame}

\begin{frame}{Proposed}
	\begin{itemize}
        \setlength\itemsep{1em}
		\item Decomposition for a large chip
		\item Place components and route for small chips in parallel
		\item Place abstract gates and route recursively
		\item Formal correctness: Reduction from SMT
		\item Formal style: Parallel/Functional code
	\end{itemize}
\end{frame}

\section[Decomposition]{}
\begin{frame}{Subcell hierarchies}
	\begin{itemize}
        \setlength\itemsep{1em}
		\item Explicit subcell hierarchies through high modularization
		\item Implicit subcell hierarchies through exlining
		\item Preserve hierarchy in compiler interfaces
	\end{itemize}
\end{frame}

\begin{frame}{High modularization}
	\begin{itemize}
        \setlength\itemsep{1em}
		\item Example of a *very* modular chip
	\end{itemize}
\end{frame}

\begin{frame}{Exlining}
	\begin{itemize}
        \setlength\itemsep{1em}
		\item Proof of concept: picorv
	\end{itemize}
\end{frame}

\section[SMT]{}
\begin{frame}{SMT2}
	\begin{itemize}
        \setlength\itemsep{1em}
		\item Reduction of a *very* common problem and witty problem to SMT
	\end{itemize}
\end{frame}

\begin{frame}{SMT2}
	\begin{itemize}
        \setlength\itemsep{1em}
		\item Show routing related problem in integer programming
	\end{itemize}
\end{frame}

\section[Process]{}

\begin{frame}{Features}
	\begin{itemize}
        \setlength\itemsep{1em}
		\item MOSFETs
		\item LDMOSFETs (High voltage) 
		\item BJTs
		\item Zener polysilicon diodes
		\item SONOS flash cells
		\item Poly silicon resistors
		\item Metal caps
	\end{itemize}
\end{frame}

\begin{frame}{Cross section}
\begin{center}
	\begin{tikzpicture}[node distance = 3cm, auto, thick,scale=0.2, every node/.style={transform shape}]
		\fill[isolationoxide] (0.0,\LowerMoreMetalTwo) rectangle (55.0,\UpperGlass);

\fill[isolationoxide] (0.0,\LowerMoreMetal) rectangle (55.0,\LowerMoreMetalTwo);

\input{tikz_process_steps/more_metal.a.tex}

\fill[white] (2.75,\UpperMoreMetal) rectangle (4.25,\LowerMoreMetalTwo);
\fill[white] (5.50,\UpperMoreMetal) rectangle (7.0,\LowerMoreMetalTwo);
\fill[white] (8.25,\UpperMoreMetal) rectangle (9.75,\LowerMoreMetalTwo);
\fill[white] (11.00,\UpperMoreMetal) rectangle (12.5,\LowerMoreMetalTwo);
\fill[white] (13.75,\UpperMoreMetal) rectangle (15.25,\LowerMoreMetalTwo);

\fill[white] (20.35,\UpperMoreMetal) rectangle (21.65,\LowerMoreMetalTwo);
\fill[white] (22.60,\UpperMoreMetal) rectangle (23.90,\LowerMoreMetalTwo);
\fill[white] (24.35,\UpperMoreMetal) rectangle (25.65,\LowerMoreMetalTwo);

\fill[white] (26.35,\UpperMoreMetal) rectangle (27.65,\LowerMoreMetalTwo);
\fill[white] (28.10,\UpperMoreMetal) rectangle (29.15,\LowerMoreMetalTwo);
\fill[white] (29.60,\UpperMoreMetal) rectangle (30.65,\LowerMoreMetalTwo);
\fill[white] (31.10,\UpperMoreMetal) rectangle (32.15,\LowerMoreMetalTwo);
\fill[white] (32.60,\UpperMoreMetal) rectangle (33.90,\LowerMoreMetalTwo);

\fill[white] (35.10,\UpperMoreMetal) rectangle (36.15,\LowerMoreMetalTwo);
\fill[white] (36.85,\UpperMoreMetal) rectangle (37.90,\LowerMoreMetalTwo);
\fill[white] (38.35,\UpperMoreMetal) rectangle (39.40,\LowerMoreMetalTwo);
\fill[white] (39.85,\UpperMoreMetal) rectangle (40.90,\LowerMoreMetalTwo);
\fill[white] (41.15,\UpperMoreMetal) rectangle (42.15,\LowerMoreMetalTwo);

\fill[white] (43.0,\UpperMoreMetal) rectangle (44.5,\LowerMoreMetalTwo);
\fill[white] (46.5,\UpperMoreMetal) rectangle (48.0,\LowerMoreMetalTwo);

\fill[white] (48.75,\UpperMoreMetal) rectangle (50.25,\LowerMoreMetalTwo);
\fill[white] (52.75,\UpperMoreMetal) rectangle (54.25,\LowerMoreMetalTwo);


\fill[metal3] (2.75,\UpperMoreMetal) rectangle (4.25,\LowerMoreMetalTwo);
\fill[metal3] (5.50,\UpperMoreMetal) rectangle (7.0,\LowerMoreMetalTwo);
\fill[metal3] (8.25,\UpperMoreMetal) rectangle (9.75,\LowerMoreMetalTwo);
\fill[metal3] (11.00,\UpperMoreMetal) rectangle (12.5,\LowerMoreMetalTwo);
\fill[metal3] (13.75,\UpperMoreMetal) rectangle (15.25,\LowerMoreMetalTwo);

\fill[metal3] (20.35,\UpperMoreMetal) rectangle (21.65,\LowerMoreMetalTwo);
\fill[metal3] (22.60,\UpperMoreMetal) rectangle (23.90,\LowerMoreMetalTwo);
\fill[metal3] (24.35,\UpperMoreMetal) rectangle (25.65,\LowerMoreMetalTwo);

\fill[metal3] (26.35,\UpperMoreMetal) rectangle (27.65,\LowerMoreMetalTwo);
\fill[metal3] (28.10,\UpperMoreMetal) rectangle (29.15,\LowerMoreMetalTwo);
\fill[metal3] (29.60,\UpperMoreMetal) rectangle (30.65,\LowerMoreMetalTwo);
\fill[metal3] (31.10,\UpperMoreMetal) rectangle (32.15,\LowerMoreMetalTwo);
\fill[metal3] (32.60,\UpperMoreMetal) rectangle (33.90,\LowerMoreMetalTwo);

\fill[metal3] (35.10,\UpperMoreMetal) rectangle (36.15,\LowerMoreMetalTwo);
\fill[metal3] (36.85,\UpperMoreMetal) rectangle (37.90,\LowerMoreMetalTwo);
\fill[metal3] (38.35,\UpperMoreMetal) rectangle (39.40,\LowerMoreMetalTwo);
\fill[metal3] (39.85,\UpperMoreMetal) rectangle (40.90,\LowerMoreMetalTwo);
\fill[metal3] (41.15,\UpperMoreMetal) rectangle (42.15,\LowerMoreMetalTwo);

\fill[metal3] (43.0,\UpperMoreMetal) rectangle (44.5,\LowerMoreMetalTwo);
\fill[metal3] (46.5,\UpperMoreMetal) rectangle (48.0,\LowerMoreMetalTwo);

\fill[metal3] (48.75,\UpperMoreMetal) rectangle (50.25,\LowerMoreMetalTwo);
\fill[metal3] (52.75,\UpperMoreMetal) rectangle (54.25,\LowerMoreMetalTwo);

% wire
\fill[metal3] ( 2.50,\LowerMoreMetalTwo) rectangle ( 4.50,\UpperMoreMetalTwo);
\fill[metal3] ( 5.25,\LowerMoreMetalTwo) rectangle ( 7.25,\UpperMoreMetalTwo);
\fill[metal3] ( 8.00,\LowerMoreMetalTwo) rectangle (10.00,\UpperMoreMetalTwo);
\fill[metal3] (10.75,\LowerMoreMetalTwo) rectangle (12.75,\UpperMoreMetalTwo);
\fill[metal3] (13.50,\LowerMoreMetalTwo) rectangle (15.50,\UpperMoreMetalTwo);

\fill[metal3] (20.00,\LowerMoreMetalTwo) rectangle (22.00,\UpperMoreMetalTwo);
\fill[metal3] (22.50,\LowerMoreMetalTwo) rectangle (24.00,\UpperMoreMetalTwo);
\fill[metal3] (24.25,\LowerMoreMetalTwo) rectangle (25.75,\UpperMoreMetalTwo);

\fill[metal3] (26.15,\LowerMoreMetalTwo) rectangle (27.75,\UpperMoreMetalTwo);
\fill[metal3] (28.00,\LowerMoreMetalTwo) rectangle (29.25,\UpperMoreMetalTwo);
\fill[metal3] (29.50,\LowerMoreMetalTwo) rectangle (30.75,\UpperMoreMetalTwo);
\fill[metal3] (30.95,\LowerMoreMetalTwo) rectangle (32.30,\UpperMoreMetalTwo);
\fill[metal3] (32.50,\LowerMoreMetalTwo) rectangle (34.25,\UpperMoreMetalTwo);

\fill[metal3] (34.75,\LowerMoreMetalTwo) rectangle (36.25,\UpperMoreMetalTwo);
\fill[metal3] (36.50,\LowerMoreMetalTwo) rectangle (38.00,\UpperMoreMetalTwo);
\fill[metal3] (38.25,\LowerMoreMetalTwo) rectangle (39.50,\UpperMoreMetalTwo);
\fill[metal3] (39.75,\LowerMoreMetalTwo) rectangle (40.95,\UpperMoreMetalTwo);
\fill[metal3] (41.10,\LowerMoreMetalTwo) rectangle (42.35,\UpperMoreMetalTwo);

\fill[metal3] (42.75,\LowerMoreMetalTwo) rectangle (44.75,\UpperMoreMetalTwo);
\fill[metal3] (46.25,\LowerMoreMetalTwo) rectangle (48.25,\UpperMoreMetalTwo);

\fill[metal3] (48.50,\LowerMoreMetalTwo) rectangle (50.50,\UpperMoreMetalTwo);
\fill[metal3] (52.50,\LowerMoreMetalTwo) rectangle (54.50,\UpperMoreMetalTwo);



\fill[white] ( 2.65,\UpperMoreMetalTwo) rectangle ( 4.35,\UpperGlass);
\fill[white] ( 5.35,\UpperMoreMetalTwo) rectangle ( 7.15,\UpperGlass);
\fill[white] ( 8.10,\UpperMoreMetalTwo) rectangle ( 9.90,\UpperGlass);
\fill[white] (10.85,\UpperMoreMetalTwo) rectangle (12.65,\UpperGlass);
\fill[white] (13.60,\UpperMoreMetalTwo) rectangle (15.40,\UpperGlass);

\fill[white] (20.10,\UpperMoreMetalTwo) rectangle (21.90,\UpperGlass);
\fill[white] (22.60,\UpperMoreMetalTwo) rectangle (23.90,\UpperGlass);
\fill[white] (24.35,\UpperMoreMetalTwo) rectangle (25.65,\UpperGlass);

\fill[white] (26.25,\UpperMoreMetalTwo) rectangle (27.65,\UpperGlass);
\fill[white] (28.10,\UpperMoreMetalTwo) rectangle (29.15,\UpperGlass);
\fill[white] (29.60,\UpperMoreMetalTwo) rectangle (30.65,\UpperGlass);
\fill[white] (31.10,\UpperMoreMetalTwo) rectangle (32.20,\UpperGlass);
\fill[white] (32.60,\UpperMoreMetalTwo) rectangle (34.15,\UpperGlass);

\fill[white] (34.85,\UpperMoreMetalTwo) rectangle (36.15,\UpperGlass);
\fill[white] (36.60,\UpperMoreMetalTwo) rectangle (37.90,\UpperGlass);
\fill[white] (38.35,\UpperMoreMetalTwo) rectangle (39.40,\UpperGlass);
\fill[white] (39.85,\UpperMoreMetalTwo) rectangle (40.85,\UpperGlass);
\fill[white] (41.20,\UpperMoreMetalTwo) rectangle (42.25,\UpperGlass);

\fill[white] (43.00,\UpperMoreMetalTwo) rectangle (44.50,\UpperGlass);
\fill[white] (46.50,\UpperMoreMetalTwo) rectangle (48.00,\UpperGlass);

\fill[white] (48.75,\UpperMoreMetalTwo) rectangle (50.25,\UpperGlass);
\fill[white] (52.75,\UpperMoreMetalTwo) rectangle (54.25,\UpperGlass);


		\node at (5,-0.5) {\textbf{\huge{PMOS}}};
		\node at (13,-0.5) {\textbf{\huge{NMOS}}};
		\node at (22,-0.5) {\textbf{\huge{SONOS flash cell (PMOS)}}};
		\node at (30,-0.5) {\textbf{\huge{NPN BJT}}};
		\node at (38,-0.5) {\textbf{\huge{PNP BJT}}};
		\node at (46,-0.5) {\textbf{\huge{Polysilicon diode}}};
		\node at (52,-0.5) {\textbf{\huge{Polyresistor}}};
	\end{tikzpicture}
\end{center}
\end{frame}

\end{document}
